% !!! 如果要编译此文件,请确定已经安装了 elegantnote。 对于 texlive 用户,可以使用
% tlmgr install elegantnote
% 进行安装
\documentclass[blue,normal,cn]{elegantnote}
%\usepackage{lstlisting}
\newfontfamily\courier{Courier New}
\lstset{linewidth=1.1\textwidth,
	numbers=left,
	basicstyle=\small\courier,
	numberstyle=\tiny\courier,
	keywordstyle=\color{blue}\courier,
	commentstyle=\it\color[cmyk]{1,0,1,0}\courier, 
	stringstyle=\it\color[RGB]{128,0,0}\courier,
	frame=single,
	backgroundcolor=\color[RGB]{245,245,244},
	breaklines,
	extendedchars=false, 
	xleftmargin=2em,xrightmargin=2em, aboveskip=1em,
	tabsize=4, 
	showspaces=false
	basicstyle=\small\courier
}
\title{C 语言语法分析程序的设计与实现}
\version{$\epsilon$}
\date{\today}
\author{2017211305班\ 2017211240\ 于海鑫}

\begin{document}
\maketitle
本文档为编译原理课程实验``词法分析程序的设计与实现''的实验报告。

\section{概述}
\subsection{实验内容}

构建一个 C语言的词法分析器,该分析器可以完成如下任务:

\begin{itemize}
	\item 可以识别出用 C语言编写的源程序中的每个单词符号,并按照记号的形式输出每个单词符号。
	\item 跳过源程序中的注释。
	\item 统计源程序的行数,各类单词的个数,字符总数,并输出统计结果。
	\item 检查程序中存在的词法错误,并报告错误的所在位置。
	\item 通过对错误进行恢复,实现只需一次扫描即可报告程序中存在的全部语法错误。
\end{itemize}

\subsection{实验环境}
操作系统:Windows 10
编程语言:C++
IDE: CLion

\section{程序设计说明}

在不使用 flex 等 Lexer Generator 生成代码时,
我们实现词法分析程序主要就是根据语言的规范,手动写出对应的自动机。
C语言的完整 token 定义可以在其规范中的 6.4 节 
Lexical elements 内找到。本次试验中使用的 C 规范为 
ISO C N2176 草案,并根据自身状况对于标准内的定义进行了些许简化,
主要变动如下:

\begin{itemize}
	\item 不具体解析预处理器相关的内容,实际上在当前市面上主流的开源 
	C 编译器内,词法分析与预处理也是两个不同的阶段。其余的主流语言甚至
	没有 C 语言这种简单粗暴的宏机制,因此我们只将预处理相关的部分构成
	为一个单一的 PreprocessinDirective Token,而不是将其按照 ISO 
	C 规范内要求的那样按照 token 拆分甚至进行替换。
	\item 处理 Identifier 时,不识别 Universal Character Name
	\item 将 FloatingConstant 与 IntegerConstant 合并为一类 
	Token,命名为 NumericConstant, 这样我们就可以使用 C++ 的
	库函数较为方便的判断一组字符是否为 NumericConstant。
	\item 在处理字符串时,不会处理字符串内出现的换行行为,该部分与
	 C++ 中的 Raw String 类似,同时我们也不会将多个相邻的字符串
	合并为一个字符串。
	\item 在尝试构造 NumericConstant 时,我们会尽可能多的吞掉允
	许的字符,一旦处理失败,我们会生成一个 NumericConstantWithError 
	Token,以此标注对应的错误。如此设计与实际错误发生的情况比较贴合,
	可以有效查找出因为错误插入字符而造成的错误。
	\item 在遇到不识别的字符时,生成一个 Unkown Token,以此表示一个错误,
	同时吞掉不识别的字符,以此使得词法分析可以继续。
\end{itemize}

\subsection{模块划分}

\subsubsection{FileWrapper}

该类是对文件的进一步抽象化,以此提供我们所需的记录字符的位置,
统计字符总数,查看后续字符等功能。其内部实现为一个 ifstream 
以及一个 双端队列,类的定义如下:

\begin{lstlisting}[language=C++]
class FileWrapper {
    public:
    explicit FileWrapper(std::string fileName);
    char getNextChar();
    char peekChar(size_t const offset);
    void eatChars(size_t const num);
    bool eof();
    [[nodiscard]] size_t getCount() const;
    [[nodiscard]] size_t getLineCount() const;
    [[nodiscard]] std::string getName() const;
    friend std::ostream& operator<<(std::ostream& os, const FileWrapper & fileWrapper);
    TokenLocation getLocation();
private:
    void read();

    std::string fileName_;
    std::ifstream sourceFile_;
    std::deque<char> buffer_;
    bool eof_;
    size_t line_;
    size_t column_;
    size_t count_;
};
\end{lstlisting}

各个函数的定义如下:

\paragraph{explicit FileWrapper(std::string fileName)}
构造函数,根据文件名打开对应的文件。

\paragraph{char getNextChar()}
获取下一个字符,并更新对应的位置信息。

\paragraph{char peekChar(size\_t const offset)}
查看后续的字符。

\paragraph{void eatChars(size\_t const num)}
吞掉 n 个字符。

\paragraph{bool eof()}
判断文件是否结束。

\paragraph{[[nodiscard]] size\_t getCount() const}
获取当前字符计数。

\paragraph{[[nodiscard]] size\_t getLineCount() const}
获取当前行数。

\paragraph{[[nodiscard]] std::string getName() const}
获取文件名。

\paragraph{std::ostream\& operator<<(std::ostream \& os, const FileWrapper \& fileWrapper)}
输出调试信息。

\paragraph{TokenLocation getLocation()}
获取当前字符的位置。

\paragraph{void read()}
从输入流读取一个字符到队列内。

\subsubsection{TokenLocation}

用以表示 Token 起始字符的位置,定义如下:

\begin{lstlisting}[language=C++]
class TokenLocation {
    public:
        TokenLocation();
        TokenLocation(std::string file, size_t col, size_t line);
        [[nodiscard]] size_t getColCount() const;
        [[nodiscard]] std::string toString() const;
    private:
        std::string fileName;
        size_t colCount;
        size_t lineCount;
    };
\end{lstlisting}

\subsubsection{TokenType}

用以表示 Token 的类型,定义如下:

\begin{lstlisting}[language=C++]
enum class TokenType {
#define TOKEN_TYPE(X, Y) X,
#include <def/TokenType.def>
#undef TOKEN_TYPE
};
\end{lstlisting}

\subsubsection{KeyWord}

用以表示关键字的类型,定义如下:

\begin{lstlisting}[language=C++]
enum class KeyWord {
#define KEYWORD(X, Y) X,
#include <def/Keyword.def>
#undef KEYWORD
};
\end{lstlisting}

\subsection{Punctuator}

用以表示符号的类型,定义如下:

\begin{lstlisting}[language=C++]
enum class Punctuator {
#define PUNCTUATOR(X, Y) X,
#include <def/Punctuator.def>
#undef PUNCTUATOR
};
\end{lstlisting}

\subsection{Token}

用以保存 Token 的信息,定义如下:

\begin{lstlisting}[language=C++]
class Token {
    public:
        Token(TokenType tokenType, TokenLocation tokenLocation, KeyWord keyWord, Punctuator punctuator,
              std::string literalValue);
        Token(TokenType tokenType, const TokenLocation& tokenLocation, KeyWord keyWord);
        Token(TokenType tokenType, TokenLocation tokenLocation, Punctuator punctuator);
        Token(TokenType tokenType, TokenLocation tokenLocation, std::string literalValue);
        Token();
        [[nodiscard]] TokenType getTokenType() const;
        [[nodiscard]] TokenLocation getTokenLocation() const;
        [[nodiscard]] KeyWord getKeyWord() const;
        [[nodiscard]] Punctuator getPunctuator() const;
        [[nodiscard]] std::string getLiteralValue() const;
        [[nodiscard]] std::string toString() const;
        // For debug.
        friend std::ostream& operator<<(std::ostream& os, const Token & token);
    private:
        TokenType tokenType_;
        TokenLocation tokenLocation_;
        KeyWord keyWord_;
        Punctuator punctuator_;
        std::string  literalValue_;
    };
\end{lstlisting}


\subsection{Lexer}

词法分析器的主体部分,在这个类内完成 Token 的生成。类的定义如下:

\begin{lstlisting}[language=C++]
class Lexer {
public:
    explicit Lexer(const std::string& fileName);
    [[nodiscard]] Token getToken() const;
    Token getNextToken();
    [[nodiscard]] size_t getCount() const;
    [[nodiscard]] size_t getLineCount() const;
    [[nodiscard]] std::string getSrcName() const;
private:
    void skipLineComment();
    void skipBlockComment();
    Token getNextNumericToken();
    Token getNextStringLiterialToken();
    Token getNextCharacterConstantToken();
    Token getNextIdentifierToken();
    Token getNextPreprocessingDirectiveToken();
    FileWrapper fileWrapper_;
    TokenLocation currentLocation_;
    Token token_;
    std::string tokenBuffer_;
};
\end{lstlisting}

其中 getNextToken 为词法分析的核心函数,在这个函数内,
我们根据首字符的不同情况,返回不同的 token,并将之保存到类内部的当前 token 内。

\subsubsection{Counter}

计数器,对各类别 Token 出现的个数进行计数并输出。定义如下:

\begin{lstlisting}[language=C++]
class Counter {
public:
    void update(const Token& token);
    friend std::ostream& operator<<(std::ostream& os, const Counter & counter);
private:
#define TOKEN_TYPE(X, Y) size_t count_##X;
#include <def/TokenType.def>
#undef TOKEN_TYPE
};
\end{lstlisting}

\subsubsection{main}
主函数,以上述的各个模块为基础,实现试验要求。

\section{测试}
\subsection{完全正确的程序一例}
\begin{lstlisting}[language=C]
#include <stdio.h>

#define TEST(X, Y, format,  ...) \
if(X != Y) { \
    fprintf(stderr, format __VA_OPT__(,) __VA_ARGS__ ); \
} 
    
// Main Function
int main(void) {
    /* Here
    * is
    * a comment */
    unsigned long long a = 1ULL;
    double b = .24e+10f;
    uint64_t c = a ++;
    a >>= 2;
    return 0;
}
\end{lstlisting}

输出如下:

\begin{lstlisting}
PS D:\Projects\cLex\cmake-build-debug> .\cLex.exe .\right.c
Tokens:
Type: PreprocessingDirective
Location: .\right.c:1:1
Value: include <stdio.h>

Type: PreprocessingDirective
Location: .\right.c:3:1
Value: define TEST(X, Y, format,  ...) \
if(X != Y) { \
    fprintf(stderr, format __VA_OPT__(,) __VA_ARGS__ ); \
}

Type: KeyWord
Location: .\right.c:9:1
Value: int

Type: Identifier
Location: .\right.c:9:5
Value: main

Type: Punctuator
Location: .\right.c:9:9
Value: (

Type: KeyWord
Location: .\right.c:9:10
Value: void

Type: Punctuator
Location: .\right.c:9:14
Value: )

Type: Punctuator
Location: .\right.c:9:16
Value: {

Type: KeyWord
Location: .\right.c:13:5
Value: unsigned

Type: KeyWord
Location: .\right.c:13:14
Value: long

Type: KeyWord
Location: .\right.c:13:19
Value: long

Type: Identifier
Location: .\right.c:13:24
Value: a

Type: Punctuator
Location: .\right.c:13:26
Value: =

Type: NumericConstant
Location: .\right.c:13:28
Value: 1ULL

Type: Punctuator
Location: .\right.c:13:32
Value: ;

Type: KeyWord
Location: .\right.c:14:5
Value: double

Type: Identifier
Location: .\right.c:14:12
Value: b

Type: Punctuator
Location: .\right.c:14:14
Value: =

Type: NumericConstant
Location: .\right.c:14:16
Value: .24e+10f

Type: Punctuator
Location: .\right.c:14:24
Value: ;

Type: Identifier
Location: .\right.c:15:5
Value: uint64_t

Type: Identifier
Location: .\right.c:15:14
Value: c

Type: Punctuator
Location: .\right.c:15:16
Value: =

Type: Identifier
Location: .\right.c:15:18
Value: a

Type: Punctuator
Location: .\right.c:15:20
Value: ++

Type: Punctuator
Location: .\right.c:15:22
Value: ;

Type: Identifier
Location: .\right.c:16:5
Value: a

Type: Punctuator
Location: .\right.c:16:7
Value: >>=

Type: NumericConstant
Location: .\right.c:16:11
Value: 2

Type: Punctuator
Location: .\right.c:16:12
Value: ;

Type: KeyWord
Location: .\right.c:17:5
Value: return

Type: NumericConstant
Location: .\right.c:17:12
Value: 0

Type: Punctuator
Location: .\right.c:17:13
Value: ;

Type: Punctuator
Location: .\right.c:18:1
Value: }

File info:
File name:      .\right.c
Total chars:    319
Total lines:    18
KeyWord:        7
Identifier:     7
NumericConstant:        4
NumericConstant(Error Detected):        0
CharacterConstant:      0
StringLiteral:  0
Punctuator:     14
PreprocessingDirective: 2
EndOfFile:      0
Unknown:        0
\end{lstlisting}

\subsection{出错程序一例}

\begin{lstlisting}[language=C]
#include <stdio.h>

int main() {
    int 2ch;
    `
    double d = 0x14ge+f;
    return 1;
}
\end{lstlisting}

结果如下:

\begin{lstlisting}
PS D:\Projects\cLex\cmake-build-debug> .\cLex.exe .\wrong.c
Tokens:
Type: PreprocessingDirective
Location: .\wrong.c:1:1
Value: include <stdio.h>

Type: KeyWord
Location: .\wrong.c:3:1
Value: int

Type: Identifier
Location: .\wrong.c:3:5
Value: main

Type: Punctuator
Location: .\wrong.c:3:9
Value: (

Type: Punctuator
Location: .\wrong.c:3:10
Value: )

Type: Punctuator
Location: .\wrong.c:3:12
Value: {

Type: KeyWord
Location: .\wrong.c:4:5
Value: int

Type: NumericConstant(Error Detected)
Location: .\wrong.c:4:9
Value: 2c

Type: Identifier
Location: .\wrong.c:4:11
Value: h

Type: Punctuator
Location: .\wrong.c:4:12
Value: ;

Type: Unknown
Location: .\wrong.c:5:5
Value: `
    
Type: KeyWord
Location: .\wrong.c:6:5
Value: double

Type: Identifier
Location: .\wrong.c:6:12
Value: d
 
Type: Punctuator
Location: .\wrong.c:6:14
Value: =

Type: NumericConstant
Location: .\wrong.c:6:16
Value: 0x14

Type: Identifier
Location: .\wrong.c:6:20
Value: ge

Type: Punctuator
Location: .\wrong.c:6:22
Value: +

Type: Identifier
Location: .\wrong.c:6:23
Value: f

Type: Punctuator
Location: .\wrong.c:6:24
Value: ;

Type: KeyWord
Location: .\wrong.c:7:5
Value: return

Type: NumericConstant
Location: .\wrong.c:7:12
Value: 1

Type: Punctuator
Location: .\wrong.c:7:13
Value: ;

Type: Punctuator
Location: .\wrong.c:8:1
Value: }
    
File info:
File name:      .\wrong.c
Total chars:    92
Total lines:    8
KeyWord:        4
Identifier:     5
NumericConstant:        2
NumericConstant(Error Detected):        1
CharacterConstant:      0
StringLiteral:  0
Punctuator:     9
PreprocessingDirective: 1
EndOfFile:      0
Unknown:        1
\end{lstlisting}

\end{document}